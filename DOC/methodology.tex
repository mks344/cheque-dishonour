\subsection{Data description}
\label{sec:data-description}
State selection based on an initial sample check of 100,000 cases.
\begin{itemize}
\item Remove States where we were unable to download sample data (this is because of data quality of the e-courts database);
\item Remove States that do not classify any cases as relating to the Negotiable Instruments Act;
\item Remove States where < 2\% of the final orders (as a proportion of total NI cases) are machine readable and in English.
\end{itemize}

Sample description:
\begin{itemize}
\item We attempted to collect a random sample of 5,00,000 cases from
  10 States and Union Territories filed between 2014/01/01 and
  2018/12/31. We were able to successfully get data for 4,68,855
  cases.
\item We removed cases from District and Sessions courts, since we are
  only interested in original matters. After this filter we are left
  with 3,63,762 cases
\item First, we used regular expressions to check whether the ``Act
  Name'' field references the NI Act.
\item Then, we used regular expressions to search for references to
  the NI Act in the texts of the interim orders and final orders,
  where available.
\item Using the aforementioned method, we got a total of 48,191 NI Act
  cases, out of these 10,693 are pending cases and 37,498 are disposed
\end{itemize}

\begin{longtable}{@{}lrrrr@{}}
\caption{Sample description}
\label{tab:sample_desc}\\
\toprule
\textbf{State} & \textbf{NI Act cases} & \textbf{\%NI Act cases} & \textbf{Non-NI Act cases} & \textbf{Total cases}\\* \midrule
\endhead
Andhra Pradesh & 2640 & 12.4 & 18567 & 21207\\
Chandigarh & 731 & 34.9 & 1364 & 2095\\
Delhi & 5211 & 26.1 & 14742 & 19953\\
Goa & 399 & 12.8 & 2713 & 3112\\
Gujarat & 6756 & 12.0 & 49476 & 56232\\
Haryana & 5326 & 13.7 & 33542 & 38868\\
Himachal Pradesh & 1166 & 8.9 & 11948 & 13114\\
Karnataka & 11195 & 12.9 & 75807 & 87002\\
Maharashtra & 8880 & 9.7 & 82673 & 91553\\
Punjab & 5887 & 19.2 & 24739 & 30626\\
\textbf{Total} & \textbf{48191} & \textbf{13.2} & \textbf{315571} & \textbf{363762}\\* \bottomrule
\end{longtable}

\subsection{Approach to analysis}
\label{sec:approach-analysis}

\subsubsection{Text-mining}
\label{sec:text-mining}

\begin{longtable}{@{}lrrrrrr@{}}
\caption{Number of cases in which we were able to identify characteristics of interest}
\label{tab:case_chars}\\
\textbf{State} & \textbf{Non Appearance} & \textbf{Summons Trial} & \textbf{Mediation} & \textbf{Jurisdiction Issue} & \textbf{Multiplicity} & \textbf{Total} \\
\endhead
Andhra Pradesh & 1674 & 814 & 260 & 210 & 124 & 2640 \\
Chandigarh & 701 & 408 & 204 & 278 & 53 & 731 \\
Delhi & 2062 & 1359 & 1026 & 1045 & 208 & 5211 \\
Goa & 381 & 78 & 90 & 33 & 18 & 399 \\
Gujarat & 5125 & 391 & 402 & 3059 & 107 & 6756 \\
Haryana & 5214 & 3376 & 1731 & 2109 & 540 & 5326 \\
Himachal Pradesh & 864 & 351 & 506 & 299 & 33 & 1166 \\
Karnataka & 5850 & 2002 & 1846 & 953 & 410 & 11195 \\
Maharashtra & 5933 & 963 & 2546 & 3831 & 135 & 8880 \\
Punjab & 5821 & 3134 & 2100 & 2281 & 382 & 5887 \\
\textbf{Total} & \textbf{33625} & \textbf{12876} & \textbf{10711} & \textbf{14098} & \textbf{2010} & \textbf{48191}
\end{longtable}

Explain that this is likely an underestimate because orders cannot be parsed in approx 36\% of the cases.

\subsubsection{Regression model specifications}
\label{sec:model-selection}
To examine the impact of case characteristics on case duration, we rely on two different measures: (1) Duration of the case (in days) and (2) Number of hearings required to dispose of. We perform a multivariate analysis: we regress the performance indicators (i.e., duration of the case and the number of hearings required to dispose of the case) on several potential explanatory variables. We also control for State and year fixed effects. For each performance measure, we estimate the following fixed effect regression model:

\begin{equation}\label{eq:fe1}
\begin{split}
Duration_i \ & or \ Number \ of \ hearings_i \\
& = \beta_1 \ D_1(Non-appearance_i) + \beta_2 \ D_2(Jurisdiction \ Issue_i) + \beta_3 \ D_3(Mediation_i) \\
& + \beta_4 \ D_4(Multiplicity_i) + \beta_5 \ D_5(Summons_i) + \beta_6 \ D_6(Contested_i) \\
&  + \alpha_s + Y_t + \epsilon_{ist}
\end{split}
\end{equation}

where $D_1$ is a dummy variable equal to 1 if the accused was not available / out of station/absconding or absent due to other reasons, $D_2$ is a dummy variable equal to 1 if the case refers to jurisdiction issues, $D_3$ is a dummy variable which takes the value 1 if the case was referred for mediation, $D_4$ is a dummy variable equal to 1 if the case suffers from a multiplicity of proceedings, $D_5$ is a dummy variable which takes the value 1 if the case is marked as summons trial, and $D_6$ is a dummy variable equal to 1 if the case was contested. We use the State and year dummies ($\alpha_s$ and $Y_t$ respectively) to control for the macro-economic reforms and environment.


%%% Local Variables:
%%% mode: latex
%%% TeX-master: "paper_chequeDishonour"
%%% End:


%%% Local Variables:
%%% mode: latex
%%% TeX-master: "paper_chequeDishonour"
%%% End:
