We find that most of intended targets of the amicus curiae's and Supreme Court's proposals do affect the duration of \gls{ni} cases. However, cases referred to mediation take significantly longer. Therefore, referring more cases to mediation is unlikely to reduce delays in \gls{ni} cases; in fact this is likely to have the opposite effect. At the same time, we find that cases which are contested take fewer days to dispose even though they require more hearings. This indicates that judges might be more efficient at resolving these cases than mediation forums. As a corollary, increasing the number of special \gls{ni} courts might have the intended effect of increasing disposal rates and reducing pendency.

% \subsection{Policy implications}
% \label{sec:policy-implications}


\subsection{Future research}
\label{sec:future-research}

\begin{enumerate}
\item Roots of delays --- what kinds of cases, reasons for issuing cheques, types of litigants, geographical factors?
\item Effects of different laws on volume and nature of cheque-dishonour cases
\end{enumerate}


%%% Local Variables:
%%% mode: latex
%%% TeX-master: "paper_chequeDishonour"
%%% End:
