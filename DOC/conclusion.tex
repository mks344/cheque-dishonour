The Amici Curiae and Supreme Court have given the following recommendations to reduce delays and pendency of cheque-dishonour cases:
\begin{enumerate}
\item Increase use pre- and post-summons mediation
\item Expedite service of summons to reduce absconsion
\item Mandate plausible defence before conversion from summary to summons trial
\item Summon witnesses only when the accused presents a defence
\item High Courts should prepare schemes to dispose cases pending for 2-4, 4-6, 6+ years
\item Address multiplicity of proceedings
\item The magistrate should record reasons for conversion of summary to summons trial
\item Explore setting up specialised courts
\item Judicially settle the position of \S~202 CrPC
\item Jurisdictional issues
\end{enumerate}

We have been able to test the likely effect of 5 of the targets of these interventions viz --- trials being run as summons trials, non-appearance by the accused, jurisdictional issues, reference to mediation, and multiplicity of proceedings. We find that most of the intended targets of the Amici Curiae's and Supreme Court's proposals do affect the duration of \gls{ni} cases. So any interventions that can reduce the proportion of cases being converted into summons trials, prevent cases from being filed in the wrong jurisdiction, ensure the presence of the accused, and reduce the multiplicity of proceedings will reduce delays in these cases and will also reduce the number of hearings required to dispose them.

However, we find that cases referred to mediation take significantly longer to dispose. We find that referring more \gls{ni} cases to mediation can increase delays and burden courts even more. Further, there appear to be issues with the mediation process itself because not many cases return to the courts for adjudication due to the mediation failing.

At the same time, we find that contested cases take fewer days to dispose even though they require more hearings. This indicates that judges might be more efficient at resolving these cases than mediation forums. As a corollary, increasing the number of special \gls{ni} courts might have the intended effect of increasing disposal rates and reducing pendency.

Our research attempts to quantify the effects of different characteristics of \gls{ni} cases on the duration. We rely on texts of judgments and orders for much of this analysis. This approach can be improved upon courts institute systems to produce more granular data on cases. One gap in our analysis, for example, is that we cannot reliably identify why the cheque in question was issued. Further, where orders are not available, our ability to identify the characteristics of interest are limited. That is also one of the reasons we had to omit many of the states from our analysis.

Policy-making and administration are research-dependent processes. Better research requires better data. Therefore, courts should record better data about case characteristics and case flow. This will enable researchers and courts alike to better identify bottlenecks in existing processes and problematic provisions in laws. Courts, in turn, can use these analyses to improve procedures and administration and ensure better justice delivery.

%%% Local Variables:
%%% mode: latex
%%% TeX-master: "paper_chequeDishonour"
%%% End: