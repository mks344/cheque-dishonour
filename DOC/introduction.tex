India has a slow judiciary - courts are clogged with large backlogs. As of 8th December 2021, over forty-one million cases were pending across district courts.\footcite{njdg2021} One reason for the large burden on courts is believed to be the alleged large share of Negotiable Instrument Act cases.\footnote{In particular, s 138 of the Negotiable Instruments Act (Dishonour of cheque for insufficiency, etc., of funds in the account). Act 26 of 1881} As per the Law Commission of India, they represent 6.5\% and 7.8\% of all institutions and pendency in Indian courts, respectively.\footcite{lci2014_arrears} As per one order of the Supreme Court of India, they reflect more than 15\% of all criminal cases in the District Courts.\footcite{sc2020_makwanavstate} As per another order, they constitute 30\% of the total pendency in courts.\footcite{sc2020_138}

Given the disagreement, any acceptance among policymakers that Negotiable Instrument Act cases (\textit{cheque dishonour cases} hereinafter) are the reason for India's slow judiciary is likely to lead to misguided solutions. The misrepresentation leads to diverting resources from other routes of judicial reform. Thus, one requires a more detailed analysis of the proportion of such cases, their causes, timelines, etc. This information can help judges and policymakers better target interventions.

To this end, in 2020, the Supreme Court of India took on board a suo-motu case concerning the “expeditious trial of cases under section 138 of the Negotiable Instruments Act 1881”.\footcite{sc2020_138} Among other directions, the court (i) appointed a Committee of Experts (CoE)\footnote{Headed by Hon’ble Mr Justice RC Chavan, former Judge of the Bombay High Court.} and (ii) appointed Amici Curiae to assist the court.\footnote{Mr Sidharth Luthra (Sr Advocate) and Mr K Parameshwar (Advocate).} This is is a series of interventions targeted to understand the cheque dishonour cases in India. The Law Commission of India (2008), by relying on newspaper reports concerning the proportion of cheque dishonour cases, recommended setting up Fast Track Magisterial Courts.\footcite{lci2008_138, bhan2015_placing} However, it did not define a \textit{fast-track courts} or give guidance concerning how and where they would operate. Daksh (2018) examined the behaviour of cheque dishonour cases in the Indian courts by looking at over 67000 cases. In most cases, they found that resolution is delayed well beyond statutorily prescribed timelines and that certain banks and financial institutions are frequent complainants.

In 2020, the Amici appointed by the Supreme Court also made recommendations to expedite the disposal of cases. This included (i) increasing the use of pre- and post-summons mediation, (ii) expediting the service of summons to reduce absconsion, (iii) addressing the multiplicity of proceedings, etc.\footcite{amicus2020_submission} This study, based on orders of the Supreme Court and the report of the Amici Curiae, attempts to assess the effect of suggested interventions of the expeditious trial of cases under section 138 of the Negotiable Instruments Act. In particular, it examines:

\begin{enumerate}
 \item What are the determinants of case duration in NI Act cases?
 \item 
\end{enumerate}

The rest of this paper is organised as follows: after this introduction \cref{sec:history} elaborates the history of cheque dishonour provisions in India and their relation with judicial delays. \Cref{sec:methodology} describes the methodology and \cref{sec:results} presents the results thereof. Finally, \cref{sec:conclusion} concludes the paper and presents the way forward.

\section{Cheque dishonour and judicial delays}
\label{sec:history}

\begin{itemize}
 \item What is section 138
 \begin{itemize}
 \item Legal provision
 \item History
 \item Why was it added / amended
 \end{itemize}
 \item Explaining \textit{important} parts of a trial
 \item What have people said till now (repetition of introduction with details)
 \item Supreme court and law commission recommendations
\end{itemize}

%%% Local Variables:
%%% mode: latex
%%% TeX-master: "paper_chequeDishonour"
%%% End: