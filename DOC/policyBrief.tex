\documentclass[12pt,a4paper]{article}

\usepackage[margin=1in]{geometry}
\usepackage[english]{babel}
\usepackage[T1]{fontenc}
\usepackage[utf8]{inputenc}

\usepackage{longtable}
\usepackage{booktabs}
\usepackage{array}
\usepackage{multirow}
\usepackage{graphicx}
\usepackage{enumitem}
\usepackage[parfill]{parskip}
\usepackage[title,titletoc]{appendix}
\usepackage{amsmath}
\usepackage{fnpos}
\usepackage{geometry}
\usepackage{csquotes}
\usepackage{tikz}
\usepackage{pdflscape}

\usepackage[hyphens]{url}

\usepackage[backend=biber,url=false,style=authoryear,sorting=nyt,bibstyle=numeric]{biblatex}
\addbibresource{chequeBounce.bib}

\usepackage[acronym,nonumberlist,nomain]{glossaries}
\makeglossary
\loadglsentries{acronyms.tex}

\AtBeginEnvironment{quote}{\smaller}
\makeFNbottom
\geometry{margin=1in}
%\renewcommand{\baselinestretch}{1.25}
\setlength{\parskip}{1em}
\setlist{nosep}

\newcommand{\floattabu}[1]{
\vspace*{0.2in}
{\footnotesize
#1
}
\vspace*{0.2in}}

\usepackage[hidelinks,breaklinks]{hyperref}
\usepackage{cleveref}

\title{Characterising cheque dishonour cases in India: Causes for delays and policy implications}
\author{Devendra Damle\thanks{Devendra Damle is a researcher at National Institute of Public Finance and Policy.} and Jitender Madaan\thanks{Jitender Madaan is a professor at Indian Institute of Technology, Delhi.} and Karan Gulati\thanks{Karan Gulati is a researcher at the Vidhi Centre for Legal Policy.}\\ and Manish Kumar Singh\thanks{Manish Kumar Singh is a professor at Indian Institute of Technology, Rourkee.} and Nikhil Borwankar\thanks{Nikhil Borwankar is a practicing advocate.}}

\begin{document}
\maketitle

\section{Introduction}

Indian courts are clogged with large backlogs. As of 8th December 2021, over forty-one million cases were pending across district courts. One reason for the large burden on courts is believed to be the (supposed) large share of Negotiable Instruments Act, 1881 (NI Act) cases. As per the Law Commission of India, they represent 6.5\% and 7.8\% of all institutions and pendency in Indian courts, respectively. As per one order of the Supreme Court of India, they reflect more than 15\% of all criminal cases in the District Courts. As per another order, they constitute 30\% of the total pendency in courts.

Given the disagreement, any acceptance among policymakers that NI Act cases are the reason for India’s slow judiciary is likely to lead to misguided solutions. The misrepresentation could lead to diverting resources from other routes of judicial reform. There is a need for a more detailed analysis of the proportion of such cases, their causes, timelines, etc. This information can help judges and policymakers better target interventions.

In 2020, the Supreme Court of India took on board a suo-motu case concerning the “expeditious trial of cases under section 138 of the Negotiable Instruments Act 1881”. Among other directions, it appointed Amici Curiae to assist the court to study processes to expedite disposal of complaints under § 138 of the NI Act. The Amici appointed by the Supreme Court also made recommendations to expedite the disposal of cases. This included (i) increasing the use of pre-and post-summons mediation, (ii) expediting the service of summons to reduce absconsion, (iii) addressing the multiplicity of proceedings, etc. 

This study, supported by the DAKSH Centre for Law and Technology, IIT-Delhi, attempts to assess the effect of suggested interventions. It first answers essential questions regarding the volume of such cases across differing courts. It then examines the determinants of case duration as identified by the Amici Curiae and the Supreme Court. In particular, the case characteristics chosen to be analysed were based on their importance and the feasibility of finding reliable information about them in the eCourts data. 

They include – (i) the accused fails to appear before the court for at least one hearing; (ii) the case is converted to a summons trial; (iii) the case is referred to mediation; (iv) the case has jurisdictional issues and is transferred to another court, as a result; (v) the case contains a multiplicity of proceedings; and (vi) the case was contested.

Table 1 summarises the expected effects of the identified case characteristics.

\begin{longtable}{@{}lcc@{}}
\caption{Expected effects of case characteristics}
\label{tab:expected}\\
\toprule
\multicolumn{1}{c}{\multirow{2}{*}{\textbf{Case characteristic}}} & \multicolumn{2}{c}{\textbf{Expected effect}} \\ \cmidrule(l){2-3} 
\multicolumn{1}{c}{} & \textbf{Duration} & \textbf{Hearings} \\ \midrule
Non-appearance of the accused & + & + \\
Conversion to summons trial & + & + \\
Mediation & - & - \\
Jurisdictional issues & + & + \\
Multiplicity of proceedings & + & + \\
Contested cases & + & + \\ \bottomrule
\end{longtable}

\section{Methodology}

We first attempted to download a random sample of 100,000 cases filed between 1st January 2014 and 31st December 2018 across India from the eCourts database. We used regular expressions to check whether the “Act Name” field references the NI Act. This was the first level of case classification. We also used regular expressions to search for references to the NI Act, cheque bounce/dishonour, and § 138 in the interim and final order texts, where available. We tagged the cases found through this protocol as related to cheque dishonour under § 138 of the NI Act. Data availability and quality vary significantly across States. Consequently, we could not include all the States of India in the study. We identified the States to exclude from the analysis based on the following criteria:

\begin{itemize}
    \item States and Union Territories where we were unable to download sample data (this is because of data quality of the e-courts database);
    \item States and Union Territories that do not classify any cases as relating to the Negotiable Instruments Act;
    \item States and Union Territories where < 2\% of the final orders (as a proportion of total NI cases) are machine-readable and English.
\end{itemize}

Based on the criteria above, of the 29 States (including Jammu and Kashmir) and 7 Union Territories, the final sample covers 8 States and 2 Union Territories, shown in Table 2. 

For these 10 States and Union Territories, we drew another random sample of 500,000 cases filed between 1st January 2014 and 31st December 2018. We were able to download case details for 4,68,855 cases successfully. We removed cases from District and Sessions courts since these courts hear appeals. This gave a sample of 363,720 cases. As with the preliminary analysis, we used pattern-matching based on regular expressions to identify cases related to NI Act. In total, the sample has 48,149 NI Act cases. We manually checked the NI Act classification for approximately 1700 cases to check for erroneous classification. There was one instance of a false positive and no false negatives. 

Table 2 shows a summary of the sample. As shown, check dishonour cases in India represent approximately 13.2\% of courts’ workload (pending and disposed). This number could be underestimated because orders cannot be parsed in about 36\% of the cases in the sample. Some may be NI Act cases, even though the Act Name field does not say so.

{\footnotesize \begin{longtable}{@{}lrrr|r@{}}
\caption{Sample description}
\label{tab:sample_desc}\\
\toprule
\textbf{State} & \textbf{NI Act cases} & \textbf{Non-NI Act cases} & \textbf{\%NI Act cases} & \textbf{Total cases}\\ \midrule
\endhead
Andhra Pradesh & 2638 & 18567 & 12.4 & 21205\\
Chandigarh & 731 & 1364 & 34.9 & 2095\\
Delhi & 5202 & 14742 & 26.1 & 19944\\
Goa & 399 & 2713 & 12.8 & 3112\\
Gujarat & 6752 & 49476 & 12.0 & 56228\\
Haryana & 5319 & 33542 & 13.7 & 38861\\
Himachal Pradesh & 1164 & 11948 & 8.9 & 13112\\
Karnataka & 11184 & 75807 & 12.9 & 86991\\
Maharashtra & 8875 & 82673 & 9.7 & 91548\\
Punjab & 5885 & 24739 & 19.2 & 30624\\
\midrule
\textbf{Total} & \textbf{48149} & \textbf{315571} & \textbf{13.2} & \textbf{363720}\\ \bottomrule
\end{longtable}
}

Table 3 shows the number of disposed and pending cases state-wise and the median time is taken to dispose of cases. According to § 143 (3) of the NI Act, courts are required to, as far as reasonably possible, complete NI Act cases within six months, i.e., 180 days. However, as the table shows, the median time to dispose of these cases is above the recommended limit in all states.

{\footnotesize
\begin{longtable}{@{}lrrrr@{}}
 \caption{State-wise disposed cases and time taken to dispose}\label{tab:state_disposal}\\
\toprule
\textbf{State} & \multicolumn{1}{p{2.5cm}}{\textbf{Disposed cases}} & \multicolumn{1}{p{2.5cm}}{\textbf{Pending cases}} & \multicolumn{1}{p{2.5cm}}{\textbf{Median time taken}} & \multicolumn{1}{p{2.5cm}}{\textbf{Median hearings}} \\
\midrule
Andhra Pradesh & 2154 & 484 & 451 & 16 \\
Chandigarh & 703 & 28 & 453 & 9 \\
Delhi & 3910 & 1292 & 518 & 8 \\
Goa & 280 & 119 & 352 & 16 \\
Gujarat & 5183 & 1569 & 323 & 13 \\
Haryana & 4241 & 1078 & 488 & 12 \\
Himachal Pradesh & 673 & 491 & 547 & 16 \\
Karnataka & 10221 & 963 & 218 & 6 \\
Maharashtra & 5024 & 3851 & 520 & 19 \\
Punjab & 5080 & 805 & 440 & 13 \\
\midrule
\textbf{Overall} & \textbf{37469} & \textbf{10680} & \textbf{395} & \textbf{11} \\
\bottomrule
\end{longtable}}

Moreover, of the pending cases, 35\% have been pending for 2 to 3 years. There is state-wise variation in the distribution of pending cases. Hence, the recommendation of the Supreme Court that the respective High Courts find solutions suited explicitly for their States has merit.

We analysed interim and final orders using a pattern matching protocol and regular expressions to identify cases with specific characteristics to increase case duration. We supplemented this with the information reported in the case details downloaded from the eCourts database. In particular, we relied on the purpose of hearings and the nature of disposal. We then used a fixed-effects regression model to estimate the effect size of these characteristics on the total time of the case. For the empirical analysis, we only used the 37,498 disposed of cases.

\section{Result and discussion}

Table 4 shows the effect of the identified case characteristics. The State in which the case is filed has a large and statistically significant impact on the case duration. Controlling for the State-level effects, all but one of the selected case characteristics significantly increase the case duration. As with the total time to dispose of the cases, the State in which the case is filed has a large and statistically significant effect on the number of hearings. Controlling for the State-level effects, all selected case characteristics significantly increase the number of hearings required to dispose of the case.

{\footnotesize \begin{longtable}{@{}p{2.5cm}rrrrr}
 \caption{Summary of results}\label{tab:summary_results}\\
 \toprule
 \textbf{Characteristic} & \multicolumn{1}{p{2cm}}{\textbf{Number of cases}} &
 \multicolumn{1}{p{2cm}}{\textbf{As \% of NI Act cases}}
 & \multicolumn{1}{p{2cm}}{\textbf{As \% of total cases}}
 & \multicolumn{1}{p{2cm}}{\textbf{Effect on days to dispose}} &
 \multicolumn{1}{p{2cm}}{\textbf{Effect on hearings to dispose}}
 \\
 \midrule
 Non appearance of the accused & 33625 & 69.8\% & 9.2\% & +213 & +7.0 \\ \midrule
 Conversion to summons trial & 12876 & 26.7\% & 3.5\% & +111 & +7.2 \\ \midrule
 Mediation & 10711 & 22.2\% & 2.9\% & +108 & +3.3 \\ \midrule
 Jurisdictional issues & 14098 & 29.2\% & 3.9\% & +287 & +5.7 \\ \midrule
 Multiplicity of proceedings & 2010 & 4.2\% & 0.6\% & +171 & +10.0 \\ \midrule
 Contested & 8283 & 17.2\% & 2.3\% & --46 & +2.9 \\ \midrule
 Total & 48191 & 100.0\% & 13.2\% & N/A & N/A \\
 \bottomrule
 \\
 \multicolumn{6}{l}{{\footnotesize \emph{Note: `+' sign
 indicates an increase, while `-' sign indicates decrease.}}}\\
\end{longtable}
}

The delays resulting from the respective characteristics affect a significant proportion of NI Act cases. Since NI Act cases constitute 13.2\% of courts’ workload, these delays are bound to contribute to the overall delays in courts. They affect the overall pendency in the judiciary.

For example, suppose the problem of accused persons not appearing before the court is addressed. In that case, all else being equal, the total duration of over 9\% of cases would reduce by 213 days, and courts will have to allocate seven fewer hearings per case. Similarly, if jurisdictional issues are addressed, the total duration of close to 4\% of cases would reduce by 287 days, and courts will have to allocate six fewer hearings per case.

As per the National Judicial Data Grid, as of January 2022, there were 3,57,72,846 total original pending cases in subordinate courts in India. Assuming the proportions and ratios of the analysis and results hold good for the country as a whole, 13.2\% of these cases would relate to the NI Act, which would amount to 47,39,902 cases. To put the numbers in the table into context, ensuring the accused’s presence can reduce 2,32,58,415 hearings across courts in the country. Similarly, if all NI Act cases are disposed of summarily, courts can avoid 90,86,676 hearings across the country.

\section{Conclusion}

The Amici Curiae and Supreme Court gave several recommendations to reduce delays and pendency of cheque-dishonour. We have been able to test the likely effect of 6 of the targets of these interventions, viz — trials being run as summons trials, non-appearance by the accused, jurisdictional issues, reference to mediation, and multiplicity of proceedings, and impact of setting up additional NI Act courts. Table 5 summarises the expected and observed effects of the identified case characteristics.

\begin{longtable}{@{}lcc|cc@{}}
\caption{Expected and observed effects of case characteristics}
\label{tab:observed}\\
\toprule
\multicolumn{1}{c}{\multirow{2}{*}{\textbf{Case characteristic}}} & \multicolumn{2}{c}{\textbf{Expected effect}} & \multicolumn{2}{c}{\textbf{Observed effect}} \\ \cmidrule(l){2-5}
\multicolumn{1}{c}{} & \textbf{Duration} & \textbf{Hearings} & \textbf{Duration} & \textbf{Hearings}\\ \midrule
Non-appearance of the accused & + & + & + & + \\
Conversion to summons trial & + & + & + & +\\
Mediation & - & - & + & + \\
Jurisdictional issues & + & + & + & + \\
Multiplicity of proceedings & + & + & + & +\\
Contested cases & + & + & - & +\\ \bottomrule
\end{longtable}

Most of the intended targets of the Amici Curiae’s and Supreme Court’s proposals do affect the duration of NI Act cases. So, any interventions that can reduce the proportion of cases converted into summons trials, prevent cases from being filed in the wrong jurisdiction, ensure the presence of the accused, and reduce the multiplicity of proceedings will reduce delays in these cases and will also reduce the number of hearings required to dispose of them.

However, cases referred to mediation take significantly longer to dispose. Referring more NI Act cases to mediation may increase delays and burden courts even more. Further, there appear to be issues with the mediation process itself because not many cases return to the courts for adjudication due to the mediation failing. At the same time, contested cases take fewer days to dispose of even though they require more hearings. This indicates that judges might be more efficient at resolving these cases than mediation forums. As a result, increasing the number of special NI Act courts might have the intended effect of rising disposal rates and reducing pendency.

Policy-making and administration are research-dependent processes. Better research requires better data. Therefore, courts should record better data about case characteristics and case flow. This will enable researchers and courts alike to conduct better judicial impact assessments and identify bottlenecks in existing processes and problematic provisions in laws. Courts, in turn, can use these analyses to improve procedures and administration and ensure better justice delivery.

\end{document}