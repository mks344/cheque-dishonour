\documentclass[11pt,a4paper]{article}

\usepackage[margin=1.5cm]{geometry}
\usepackage[english]{babel}
\usepackage[T1]{fontenc}
\usepackage[utf8]{inputenc}

\usepackage{longtable}
\usepackage{booktabs}
\usepackage{array}
\usepackage{multirow}
\usepackage{graphicx}
\usepackage{enumitem}
\usepackage[parfill]{parskip}
\usepackage[title,titletoc]{appendix}
\usepackage{amsmath}
\usepackage{fnpos}
\usepackage{geometry}
\usepackage{csquotes}
\usepackage{tikz}
\usepackage{pdflscape}

\usepackage[hyphens]{url}

\usepackage[backend=biber,url=false,style=authoryear,sorting=nyt,bibstyle=numeric]{biblatex}
\addbibresource{chequeBounce.bib}

\usepackage[acronym,nonumberlist,nomain]{glossaries}
\makeglossary
\loadglsentries{acronyms.tex}

\AtBeginEnvironment{quote}{\smaller}
\makeFNbottom
\geometry{margin=1in}
%\renewcommand{\baselinestretch}{1.25}
\setlength{\parskip}{1em}
\setlist{nosep}

\newcommand{\floattabu}[1]{
\vspace*{0.2in}
{\footnotesize
#1
}
\vspace*{0.2in}}

\usepackage[hidelinks,breaklinks]{hyperref}
\usepackage{cleveref}

\title{\vspace{-1.5cm}Characterising cheque dishonour cases in India: Causes for delays and policy implications}
\author{Devendra Damle\thanks{Devendra Damle is a researcher at the National Institute of Public Finance and Policy.} and Jitender Madaan\thanks{Jitender Madaan is a professor at the Indian Institute of Technology, Delhi.} and Karan Gulati\thanks{Karan Gulati is a researcher at the Vidhi Centre for Legal Policy.}\\ and Manish Kumar Singh\thanks{Manish Kumar Singh is a professor at the Indian Institute of Technology, Roorkee.} and Nikhil Borwankar\thanks{Nikhil Borwankar is a practicing advocate.}}
\date{\vspace{-1cm}}

\begin{document}
\maketitle

\section{Introduction}

Indian courts are clogged with large backlogs. As of 8th December 2021, over forty-one million cases were pending across district courts. One reason for the large burden on courts is believed to be the (supposed) large share of Negotiable Instruments Act, 1881 (NI Act) cases. As per the Law Commission of India, they represent 6.5\% and 7.8\% of all institutions and pendency in Indian courts, respectively. As per one order of the Supreme Court of India, they reflect more than 15\% of all criminal cases in the District Courts. As per another order, they constitute 30\% of the total pendency in courts. Given the disagreement, any acceptance among policymakers that NI Act cases are the reason for India’s slow judiciary is likely to lead to misguided solutions. The misrepresentation could lead to diverting resources from other routes of judicial reform. There is a need for a more detailed analysis of the proportion of such cases, their causes, timelines, etc. This information can help judges and policymakers better target interventions.

In 2020, the Supreme Court of India took on board a suo-motu case concerning the \emph{``expeditious trial of cases under section 138 of the Negotiable Instruments Act 1881''}. Among other directions, it appointed Amici Curiae to assist the court to study processes to expedite disposal of complaints under § 138 of the NI Act. The Amici appointed by the Supreme Court also made several recommendations to expedite the disposal of cases, including expediting service of summons, setting up specialised courts, combining cases against the same accused into one, clarifying jurisdictional issues, encouraging more pre-trial and post-summons mediation, and limiting how many cases are converted into summons trials.\footcite{amicus2020_submission}

This study, supported by the DAKSH Centre for Law and Technology, IIT-Delhi, attempts to assess the potential effect of the proposed interventions. The study first answers essential questions regarding the volume of such cases across differing courts. It then examines the determinants of case duration by studying what effect each target of proposed interventions has on the overall time to dispose the and number of hearings required to dispose the case. The case characteristics whose impact we have chosen to examine were determined based on two factors: (1) the importance of that characteristic and (2) the feasibility of finding reliable information about it in the eCourts data. The six chosen case characteristics map onto the interventions proposed by the Supreme Court and Amici Curiae. The characteristics we chose are as follows:
\begin{enumerate}
\item The accused fails to appear before the court for at least one hearing;
\item The case is converted to a summons trial;
\item The case is referred to mediation;
\item The case has jurisdictional issues and is transferred to another court, as a result;
\item The case contains a multiplicity of proceedings --- either the dishonoured cheque was issued to satisfy multiple transactions, or multiple cheques were dishonoured;
\item The case was contested.
\end{enumerate}

These characteristics can be thought of as events in the life-cycle of a case. They are not mutually exclusive, and all or none of the events may occur in a given case. Each proposed interventions attempts to address one or more of these events in a case. We used a systematic protocol to identify states to gather case-level information reliably. We short-listed 8 states and 2 union territories, and then drew a random sample of 363,720 cases. We analysed the case data as reported by eCourts, and the texts of the orders. The results of the study are summarised below.

\section{Overview of NI Act litigation}
\label{sec:findings}

Table \ref{tab:sample_desc} shows a summary of the sample. In India, cheque dishonour cases represent approximately 13.2\% of courts' workload (pending and disposed). This number could be an underestimate because orders were either unavailable or could not be parsed in about 36\% of the cases in the sample. Some may be NI Act cases, even though the Act Name field does not say so.

{\footnotesize \begin{longtable}{@{}lrrrr|c@{}}
\caption{Sample description}
\label{tab:sample_desc}\\
\toprule
\textbf{State} & \multicolumn{1}{p{1.5cm}}{\textbf{NI Act cases}} & \multicolumn{1}{p{1.5cm}}{\textbf{Non-NI Act cases}} & \multicolumn{1}{p{1.5cm}}{\textbf{\%NI Act cases}} & \textbf{Total cases} & \multicolumn{1}{p{3cm}}{\textbf{Median days to dispose NI Act cases}}\\ \midrule
\endhead
Andhra Pradesh & 2638 & 18567 & 12.4 & 21205 & 451\\
Chandigarh & 731 & 1364 & 34.9 & 2095 & 453\\
Delhi & 5202 & 14742 & 26.1 & 19944 & 518\\
Goa & 399 & 2713 & 12.8 & 3112 & 352\\
Gujarat & 6752 & 49476 & 12.0 & 56228 & 323\\
Haryana & 5319 & 33542 & 13.7 & 38861 & 488\\
Himachal Pradesh & 1164 & 11948 & 8.9 & 13112 & 491\\
Karnataka & 11184 & 75807 & 12.9 & 86991 & 218\\
Maharashtra & 8875 & 82673 & 9.7 & 91548 & 520\\
Punjab & 5885 & 24739 & 19.2 & 30624 & 440\\
\midrule
\textbf{Total} & \textbf{48149} & \textbf{315571} & \textbf{13.2} & \textbf{363720} & \textbf{395}\\ \bottomrule
\end{longtable}
}

According to § 143 (3) of the NI Act, courts are required to, as far as reasonably possible, complete NI Act cases within six months, i.e., 180 days. However, we find that the median time to dispose of these cases is above the recommended limit in all states. Moreover, of the pending cases, 35\% have been pending for 2 to 3 years.

\section{Determinants of duration and number of hearings}

Table \ref{tab:summary_results} shows the effect of the identified case characteristics, controlling for the State-level effects. Cases where the accused fails to appear take an additional 213 days and 7 hearings to dispose, all else being equal. Similarly, cases run as summons trials take 111 days and 7.2 hearings more than summary trials. Cases with jurisdictional issues take 287 days and 5.7 hearings more to dispose than cases without these issues. Cases with a multiplicity of proceedings take an additional 171 days and 10 hearings to dispose than those without. Contested cases take fewer days to dispose than uncontested cases, even though they take more hearings.

{\footnotesize \begin{longtable}{@{}p{2.5cm}rrrrr}
 \caption{Summary of results}\label{tab:summary_results}\\
 \toprule
 \textbf{Characteristic} & \multicolumn{1}{p{2cm}}{\textbf{Number of cases}} &
 \multicolumn{1}{p{2cm}}{\textbf{As \% of NI Act cases}}
 & \multicolumn{1}{p{2cm}}{\textbf{As \% of total cases}}
 & \multicolumn{1}{p{2cm}}{\textbf{Effect on days to dispose}} &
 \multicolumn{1}{p{2cm}}{\textbf{Effect on hearings to dispose}}
 \\
 \midrule
 Non appearance of the accused & 33625 & 69.8\% & 9.2\% & +213 & +7.0 \\ \midrule
 Conversion to summons trial & 12876 & 26.7\% & 3.5\% & +111 & +7.2 \\ \midrule
 Mediation & 10711 & 22.2\% & 2.9\% & +108 & +3.3 \\ \midrule
 Jurisdictional issues & 14098 & 29.2\% & 3.9\% & +287 & +5.7 \\ \midrule
 Multiplicity of proceedings & 2010 & 4.2\% & 0.6\% & +171 & +10.0 \\ \midrule
 Contested & 8283 & 17.2\% & 2.3\% & --46 & +2.9 \\ \midrule
 Total & 48191 & 100.0\% & 13.2\% & N/A & N/A \\
 \bottomrule
 \\
 \multicolumn{6}{l}{{\footnotesize \emph{Note: `+' sign
 indicates an increase, while `-' sign indicates decrease.}}}\\
\end{longtable}
}

So, most of the intended targets of the Amici Curiae's and Supreme Court's proposals do increase the duration of NI Act cases. So, any interventions that can reduce the proportion of cases converted into summons trials, prevent cases from being filed in the wrong jurisdiction, ensure the presence of the accused, reduce the multiplicity of proceedings, and increase the rate at which cases are heard (e.g.: increasing the number of courts) will reduce delays in these cases and will also reduce the number of hearings required to dispose of them.

A striking departure from this are cases referred to mediation. These cases take 108 days more to dispose than cases not referred to mediation. Further, this likely does not save judges' time because they take on average 3.3 hearings more to dispose. So, referring more NI Act cases to mediation may increase delays and burden courts even more.

\section{Closer look at mediation}
\label{sec:closer-look-at}

The mediation process concludes in a successful settlement in a vast majority of the cases, as shown in Table \ref{tab:mediation}. The case gets sent back to the court for adjudication only in 21.4\% of instances. Read with the result on the duration of cases referred to mediation being longer than other cases; this means that the delays result from issues in the mediation process itself, and not (as the literature in other jurisdictions indicates) in whether or not the parties choose to settle. Precisely identifying issues with the mediation process requires further investigation. However, until these issues are properly identified and understood, courts referring \gls{ni} cases to mediation will likely result in greater delays and pendency.

{\footnotesize \begin{longtable}{@{}clrrr@{}}
 \caption{Outcomes of cases referred to mediation}
 \label{tab:mediation}\\
 \toprule
 \textbf{Disposal type} & \multicolumn{1}{c}{\textbf{Disposal sub-type}} & \multicolumn{1}{c}{\textbf{Total cases}} & \multicolumn{1}{c}{\textbf{Percentage}} & \multicolumn{1}{p{3cm}}{\textbf{Median duration (in days)}} \\
 \midrule \endhead
 \multirow{3}{*}{Dismissed} & other & 478 & 5.46 & 710 \\
 & settlement & 8 & 0.09 & 512 \\
 & withdrawn & 2942 & 33.58 & 491 \\
 \midrule
 \multirow{4}{*}{Disposed} & compounded & 459 & 5.24 & 638 \\
 & other & 1395 & 15.92 & 723 \\
 & settlement & 2945 & 33.62 & 527 \\
 & withdrawn & 213 & 2.43 & 462 \\
 \midrule
 Other & other & 320 & 3.65 & 752 \\
 \midrule
 \multicolumn{2}{c}{\textbf{Total}} & \textbf{8760} & \textbf{100.00} & \multicolumn{1}{l}{\textbf{-}} \\
 \bottomrule \multicolumn{5}{p{11cm}}{{\footnotesize \emph{Note:
 The total is less than the total cases referred to mediation owing to the limitations in the data on disposal type and filing/disposal dates.}}}
 \end{longtable}}

This result read with the fact that contested cases take fewer days to dispose indicate that judges might be more efficient at resolving these cases than mediation forums. As a result, increasing the number of special NI Act courts might have the intended effect of raising disposal rates and reducing pendency, rather than more mediation.

\section{Potential impact and way forward} \label{sec:impact-case-loads}

As per the \gls{njdg}, as of January 2022, there were 3,57,72,846 total original pending cases in subordinate courts in India. Assuming the proportions and ratios of our analysis and results hold good for the country as a whole, 13.2\% of these cases would relate to the \gls{ni}, which would amount to 47,39,902 cases. The delays resulting from the respective characteristics affect a significant proportion of \gls{ni} cases. Since \gls{ni} cases constitute 13.2\% of courts' workload, these delays are bound to contribute to the overall delays in courts. So referring these cases to mediation, for example, could add 34,40,884 hearings to the courts across the country (over the lifetimes of those cases).\footnote{This assumes that the proportion of the distribution of case characteristics holds good for the country as a whole.} These hearings could be avoided if the courts tried and disposed these cases. The difference is significant and highlights why it is necessary to support policy decisions with robust empirical studies. Even where this study supports the proposed recommendations, it adds value by giving estimates of the scale of the impact, thus giving a framework to prioritise the interventions.

Policy-making and administration are research-dependent processes. Better research requires better data. Therefore, courts should also institute mechanisms to record better case characteristics and case flow data. This will enable researchers and courts alike to conduct better judicial impact assessments and identify bottlenecks in existing processes and problematic provisions in laws. Courts, in turn, can use these analyses to improve procedures and administration and ensure better justice delivery.

\end{document}
